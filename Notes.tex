%%%%%%%%%%%%%%%%%%%%%%%%%%%%%%%%%%%%%%%%%
% Compact Laboratory Book
% LaTeX Template
% Version 1.0 (4/6/12)
%
% This template has been downloaded from:
% http://www.LaTeXTemplates.com
%
% Original author:
% Joan Queralt Gil (http://phobos.xtec.cat/jqueralt) using the labbook class by
% Frank Kuster (http://www.ctan.org/tex-archive/macros/latex/contrib/labbook/)
%
% License:
% CC BY-NC-SA 3.0 (http://creativecommons.org/licenses/by-nc-sa/3.0/)
%
% Important note:
% This template requires the labbook.cls file to be in the same directory as the
% .tex file. The labbook.cls file provides the necessary structure to create the
% lab book.
%
% The \lipsum[#] commands throughout this template generate dummy text
% to fill the template out. These commands should all be removed when 
% writing lab book content.
%
% HOW TO USE THIS TEMPLATE 
% Each day in the lab consists of three main things:
%
% 1. LABDAY: The first thing to put is the \labday{} command with a date in 
% curly brackets, this will make a new section showing that you are working
% on a new day.
%
% 2. EXPERIMENT/SUBEXPERIMENT: Next you need to specify what 
% experiment(s) and subexperiment(s) you are working on with a 
% \experiment{} and \subexperiment{} commands with the experiment 
% shorthand in the curly brackets. The experiment shorthand is defined in the 
% 'DEFINITION OF EXPERIMENTS' section below, this means you can 
% say \experiment{pcr} and the actual text written to the PDF will be what 
% you set the 'pcr' experiment to be. If the experiment is a one off, you can 
% just write it in the bracket without creating a shorthand. Note: if you don't 
% want to have an experiment, just leave this out and it won't be printed.
%
% 3. CONTENT: Following the experiment is the content, i.e. what progress 
% you made on the experiment that day.
%
%%%%%%%%%%%%%%%%%%%%%%%%%%%%%%%%%%%%%%%%%

%----------------------------------------------------------------------------------------
%	PACKAGES AND OTHER DOCUMENT CONFIGURATIONS
%----------------------------------------------------------------------------------------                               
\documentclass[fontsize=11pt, % Document font size
                             paper=a4, % Document paper type
                             twoside, % Shifts odd pages to the left for easier reading when printed, can be changed to oneside
                             captions=tableheading,
                             index=totoc,
                             hyperref]{labbook}
 
\usepackage[bottom=10em]{geometry} % Reduces the whitespace at the bottom of the page so more text can fit

\usepackage[english]{babel} % English language
\usepackage{lipsum} % Used for inserting dummy 'Lorem ipsum' text into the template

\usepackage[utf8]{inputenc} % Uses the utf8 input encoding
\usepackage[T1]{fontenc} % Use 8-bit encoding that has 256 glyphs

\usepackage[osf]{mathpazo} % Palatino as the main font
\linespread{1.05}\selectfont % Palatino needs some extra spacing, here 5% extra
\usepackage[scaled=.88]{beramono} % Bera-Monospace
\usepackage[scaled=.86]{berasans} % Bera Sans-Serif

\usepackage{booktabs,array} % Packages for tables

\usepackage{amsmath} % For typesetting math
\usepackage{graphicx} % Required for including images
\usepackage{etoolbox}
\usepackage[norule]{footmisc} % Removes the horizontal rule from footnotes
\usepackage{lastpage} % Counts the number of pages of the document

\usepackage[dvipsnames]{xcolor}  % Allows the definition of hex colors
\definecolor{titleblue}{rgb}{0.16,0.24,0.64} % Custom color for the title on the title page
\definecolor{linkcolor}{rgb}{0,0,0.42} % Custom color for links - dark blue at the moment

\addtokomafont{title}{\Huge\color{titleblue}} % Titles in custom blue color
\addtokomafont{chapter}{\color{OliveGreen}} % Lab dates in olive green
\addtokomafont{section}{\color{Sepia}} % Sections in sepia
\addtokomafont{pagehead}{\normalfont\sffamily\color{gray}} % Header text in gray and sans serif
\addtokomafont{caption}{\footnotesize\itshape} % Small italic font size for captions
\addtokomafont{captionlabel}{\upshape\bfseries} % Bold for caption labels
\addtokomafont{descriptionlabel}{\rmfamily}
\setcapwidth[r]{10cm} % Right align caption text
\setkomafont{footnote}{\sffamily} % Footnotes in sans serif

\deffootnote[4cm]{4cm}{1em}{\textsuperscript{\thefootnotemark}} % Indent footnotes to line up with text

\DeclareFixedFont{\textcap}{T1}{phv}{bx}{n}{1.5cm} % Font for main title: Helvetica 1.5 cm
\DeclareFixedFont{\textaut}{T1}{phv}{bx}{n}{0.8cm} % Font for author name: Helvetica 0.8 cm

\usepackage[nouppercase,headsepline]{scrpage2} % Provides headers and footers configuration
\pagestyle{scrheadings} % Print the headers and footers on all pages
\clearscrheadfoot % Clean old definitions if they exist

\automark[chapter]{chapter}
\ohead{\headmark} % Prints outer header

\setlength{\headheight}{25pt} % Makes the header take up a bit of extra space for aesthetics
\setheadsepline{.4pt} % Creates a thin rule under the header
\addtokomafont{headsepline}{\color{lightgray}} % Colors the rule under the header light gray

\ofoot[\normalfont\normalcolor{\thepage\ |\  \pageref{LastPage}}]{\normalfont\normalcolor{\thepage\ |\  \pageref{LastPage}}} % Creates an outer footer of: "current page | total pages"

% These lines make it so each new lab day directly follows the previous one i.e. does not start on a new page - comment them out to separate lab days on new pages
\makeatletter
\patchcmd{\addchap}{\if@openright\cleardoublepage\else\clearpage\fi}{\par}{}{}
\makeatother
\renewcommand*{\chapterpagestyle}{scrheadings}

% These lines make it so every figure and equation in the document is numbered consecutively rather than restarting at 1 for each lab day - comment them out to remove this behavior
\usepackage{chngcntr}
\counterwithout{figure}{labday}
\counterwithout{equation}{labday}

% Hyperlink configuration
\usepackage[
    pdfauthor={}, % Your name for the author field in the PDF
    pdftitle={Laboratory Journal}, % PDF title
    pdfsubject={}, % PDF subject
    bookmarksopen=true,
    linktocpage=true,
    urlcolor=linkcolor, % Color of URLs
    citecolor=linkcolor, % Color of citations
    linkcolor=linkcolor, % Color of links to other pages/figures
    backref=page,
    pdfpagelabels=true,
    plainpages=false,
    colorlinks=true, % Turn off all coloring by changing this to false
    bookmarks=true,
    pdfview=FitB]{hyperref}

\usepackage[stretch=10]{microtype} % Slightly tweak font spacing for aesthetics

%\setlength\parindent{0pt} % Uncomment to remove all indentation from paragraphs

\usepackage{mathrsfs, euscript} % Support command like \mathscr
% ***********************************************************
% ******************* PHYSICS HEADER ************************
% ***********************************************************
% Version 2

%-----------------------------------------------------------------------------------------
% Physics Command
%-----------------------------------------------------------------------------------------
%\renewcommand{\labelenumi}{(\alph{enumi})} % Use letters for enumerate
% \DeclareMathOperator{\Sample}{Sample}
\let\vaccent=\v % rename builtin command \v{} to \vaccent{}
\renewcommand{\v}[1]{\ensuremath{\mathbf{#1}}} % for vectors
\newcommand{\gv}[1]{\ensuremath{\mbox{\boldmath$ #1 $}}} 
% for vectors of Greek letters
\newcommand{\uv}[1]{\ensuremath{\mathbf{\hat{#1}}}} % for unit vector
\newcommand{\abs}[1]{\left| #1 \right|} % for absolute value
\newcommand{\avg}[1]{\left< #1 \right>} % for average
\let\underdot=\d % rename builtin command \d{} to \underdot{}
\renewcommand{\d}[2]{\frac{\mathrm{d} #1}{\mathrm{d} #2}} % for derivatives
\newcommand{\dd}[2]{\frac{\mathrm{d}^2 #1}{\mathrm{d} #2^2}} % for double derivatives
\newcommand{\pd}[2]{\frac{\partial #1}{\partial #2}} 
% for partial derivatives
\newcommand{\pdd}[2]{\frac{\partial^2 #1}{\partial #2^2}} 
% for double partial derivatives
\newcommand{\pdc}[3]{\left( \frac{\partial #1}{\partial #2}
 \right)_{#3}} % for thermodynamic partial derivatives
\newcommand{\ket}[1]{\left| #1 \right>} % for Dirac bras
\newcommand{\bra}[1]{\left< #1 \right|} % for Dirac kets
\newcommand{\braket}[2]{\left< #1 \vphantom{#2} \right|
 \left. #2 \vphantom{#1} \right>} % for Dirac brackets
\newcommand{\matrixel}[3]{\left< #1 \vphantom{#2#3} \right|
 #2 \left| #3 \vphantom{#1#2} \right>} % for Dirac matrix elements
\newcommand{\grad}[1]{\gv{\nabla} #1} % for gradient
\let\divsymb=\div % rename builtin command \div to \divsymb
\renewcommand{\div}[1]{\gv{\nabla} \cdot #1} % for divergence
\newcommand{\curl}[1]{\gv{\nabla} \times #1} % for curl
\let\baraccent=\= % rename builtin command \= to \baraccent
\renewcommand{\=}[1]{\stackrel{#1}{=}} % for putting numbers above =
% for operator command
\DeclareMathAlphabet{\mathsfsl}{OT1}{cmss}{m}{sl}
\newcommand{\ope}[1]{\mathsfsl{#1}}
% for commutation operation
\newcommand{\commute}[2]{\left[ #1, #2 \right]}
% for anticommutation operation
\newcommand{\anticom}[2]{\left\{ #1 , #2 \right\} }
% for trace
\newcommand{\Tr}[1]{\mathrm{Tr #1}}

%-----------------------------------------------------------------------------------------
% Physics Command End
%-----------------------------------------------------------------------------------------

% ***********************************************************
% ********************** END HEADER *************************
% ***********************************************************% external physics command support
\graphicspath{{./pic/}}% set for path of pictures
%----------------------------------------------------------------------------------------
%	DEFINITION OF EXPERIMENTS
%----------------------------------------------------------------------------------------

% Template: \newexperiment{<abbrev>}[<short form>]{<long form>}
% <abbrev> is the reference to use later in the .tex file in \experiment{}, the <short form> is only used in the table of contents and running title - it is optional, <long form> is what is printed in the lab book itself

\newexperiment{example}[Example experiment]{This is an example experiment}
\newexperiment{example2}[Example experiment 2]{This is another example experiment}
\newexperiment{example3}[Example experiment 3]{This is yet another example experiment}

\newsubexperiment{subexp_example}[Example sub-experiment]{This is an example sub-experiment}
\newsubexperiment{subexp_example2}[Example sub-experiment 2]{This is another example sub-experiment}
\newsubexperiment{subexp_example3}[Example sub-experiment 3]{This is yet another example sub-experiment}

% Paper List


%----------------------------------------------------------------------------------------

\begin{document}

%----------------------------------------------------------------------------------------
%	TITLE PAGE
%----------------------------------------------------------------------------------------

\title{\textcap{Research Notes \\[1cm]  
\textaut{\usefont{T1}{phv}{m}{sc} per aspera ad astra }}}

\author{
    \textaut{Yuan Hang}\\ \\ % Your name
}
\date{Date Compiled: \today \\ Data Began: March 24, 2017} % No date by default, add \today if you wish to include the publication date

\maketitle % Title page

\printindex
\tableofcontents % Table of contents
\newpage % Start lab look on a new page
%\listoffigures % List of figures
%\newpage % new page

\begin{addmargin}[4cm]{0cm} % Makes the text width much shorter for a compact look

\pagestyle{scrheadings} % Begin using headers

%----------------------------------------------------------------------------------------
%	LAB BOOK CONTENTS
%----------------------------------------------------------------------------------------
\labday{Friday, March 24, 2017}
\experiment{101 Formulaic Alphas\cite{Kakushadze2015}}
When I saw this title, I thought this is an introductory paper to alphas as the 101 number usually indicated. It's really surpirsing that the author really wrote down 101 alphas. This kind of article is pretty different from common academic research papers. Many proprietary statement and hidden details. It's pretty interesting that this author even cited poems in the article. Anyway, as the first article to get started, I will record main points I get and list my quesitons below:
\begin{itemize}
\item Points
\begin{enumerate}
\item
Alpha is a pretty general meaning in this article, not just the alpha like the one in CAPM. The author gives a definition and I quote, ''An alpha is a combination of mathematical expressions, computer source code, and configuration parameters that can be used, in combination with historical data, to make predictions about future movements of various financial instruments.'' There are countless alphas could be constructed.(grow superexponentially?)
\item
General speaking, alpha signals could be based on mean-reversion or momentum or mixed.
\item
A ``delay-d'' alpha means if the alpha is traded today then the construction of this alpha involves data d days ago
\item
It's alphas are traded. Some characteristics of each alpha could be calculated, like shape ratio, turnover, cps, etc, from their trading data. (Characteristics of each product)
\item
The pair-wise relationship could be characterized by covariance matrix. However, study shows that variances of each alpha, which are extracted from historical data, are relatively stable.The off-diagonal elements are the focus to estimate.
\item
Using cross-sectional regressions, returns are strongly correlated with volatility V based on their data:
\begin{equation}
R\sim V^\gamma
\end{equation}
where $\gamma\approx 0.76$. Returns have no significant dependence on the turnover $T$.
\item To explore whether turnover could explain the correlation between alphas, multifactor risk model is applied. 
\begin{equation}
\Gamma_{ij}=\xi_i^2\delta_{ij}+\sum_{l,m}\Omega_{il}\Omega_{jm}\phi_{lm}
\end{equation}
I think the author applied a seemingly fancy skill to reachthe conclusion. I am still not sure why he did a symmetric tensor combinations.
\item
Result shows that linear and bilinear of log turnover have poor explanatory power of pair-wise correlations.
\item
Above conclusion does not necessarily mean the turnover adds no value in the risk factor model. It only shows that turnover does not appear to useful in modeling pair-wise alpha correlations.
\end{enumerate}
\item Questions
\begin{enumerate}
\item 
How to construct alpha systematically is not clear. If we want to construct alphas in billions, some clever methods needed.
\item
I need to study multi-factor risk model in detail
\item
How to understand three symmetrically constructed tensors are orthogonal to each other? Intuitively I agree. However, I need to understand it mathematically. I guess the answer should be found in statistics.
\item
Whether turnover adds value via the specific risk? Proprietary methods...
\end{enumerate}
\end{itemize}
\labday{Saturday, March 25, 2017}
\experiment{Performance v. Turnover: A Story by 4,000 Alphas\cite{Kakushadze2016}}
Interesting, one of the authors of this article is Igor Tulchinsky, who is the founder of WorldQuant. WorldQuant recently launched a strong wave of advertisement in campus. Anyway, this article presented a glimpse about 4,000 real-life trading alphas which is worth having a look.
\begin{itemize}
\item Points
\begin{enumerate}
\item
Target: Address the question of turnover dependence in the context of dollar-neutral quantitative trading strategies with shorter horizons, with holding periods of roughly from intraday to about 20 trading days.
\item
Relation between CPS and turnover (empirical analysis):
\begin{equation}
C\sim\frac{1}{T}
\end{equation}
\item
For holding periods up to about 10 days, the return predominately stems from volatility:
\begin{equation}
R\sim V^X
\end{equation}
where $X=0.8\to0.85$. For holding periods above 10 days, the power $X$ drops, indicating that some additional factors might enter.
\item Data Pre-processing:
\begin{itemize}
\item
Exclude negative sharpe ratio
\item
Exclude outliers by $|X_i - median(X)|>3 MAD(X)$
\end{itemize} 
\item Analysis method: linear cross-sectional regression with the intercept:
\begin{equation}
\ln(Z_i)=a+\sum_k b_k \ln(U_{ik})+\epsilon_i
\end{equation}
\item Results:
\begin{itemize}
\item
$\ln(C)\approx -2 -\ln(T)$
\item
$\ln(C)\approx-1+\ln(S)$
\end{itemize}
\item Analysis method: Run regression for different deciles to investigate sub-structures
\item This article doesn't directly calculate daily return volatility $\sigma_i$ but obtain it via the help of Sharpe ratio $S_i$
\end{enumerate}
\item Questions
\begin{enumerate}
\item
``All alphas are for trading US equities with overnight holdings. The trading universes for individual alphas are proprietary, but in most part overlap with a typical universe of most liquid US stokes used in algorithmic trading''
\item
``There is an overlap between the universes traded by different alphas. It is not unreasonable to assume that $\Pi_i$ is approximately uniform across all alphas.''
\item
Note that
\begin{equation}
R_i^\prime \sim R_i \ \ \ \ \sigma_i^\prime \sim \sigma_i
\end{equation}
\item
Possible combinations of assets is given by $2^n$. For a given assents combination, there are also exponential possible alphas(How to see this point?). I still don't know how to harvest profit via these ephemeral alphas.
\end{enumerate}
\end{itemize}
\experiment{Statistical arbitrage in the US equities market\cite{Avellaneda2010}}
I don't read this article carefully. The core assumptions of this article is the idiosyncratic returns obey mean-reverting processes. 
Their trading signals are generated by two ways: PCA and regressing stock returns on sector Exchange Traded Funds. The data analysis methods are useful for reading.
\labday{Sunday, March 26, 2017}
\experiment{Active portfolio management\cite{grinold2000active}-Structural Risk Model}
\section{Model}
For a specific asset, structural risk model states that
\begin{equation}
r_n(t,t+1)=\sum_k \beta_{n,k}(t)f_k(t,t+1)+u_n(t,t+1)
\end{equation}
where $r_n$ is the excess return, $\beta_{n,k}$ is the k-th factor exposure, $f_k$ is the k-th factor return and $u_n$ is the specific return. Note that the factor exposures should be prior variables which could be estimated at the beginning of period. Writing above equation in vectorized form, then:
\begin{equation}
\v r =\v X \cdot \v b +\v u
\end{equation}
\section{Assumptions}
\begin{itemize}
\item
specific returns are uncorrelated with factor returns, i.e.
\begin{equation}
\mathrm{Cov}(u_n,b_k)=0
\end{equation}
\item
specific returns are uncorrelated with each other, i.e.
\begin{equation}
\mathrm{Cov}(u_n,u_m)=0,\ if \ m\neq n
\end{equation}
\end{itemize}
\section{Model Estimation}
Given exposures to the industry and risk index factors, estimate factor returns via multiple regressions, using the Fama-MacBeth procedure.\\
$X$ should be estimated from fundamental data;
\subsection{Factor Portfolios}
There is an interpretation about factor return:
\begin{equation}
\v b=(\v X^T \cdot \gv \Delta^{-1}\cdot \v X)^{-1}\cdot \v X^T\cdot \gv \Delta^{-1}\cdot \v r
\end{equation}
In component form,
\begin{equation}
b_k=\sum_n c_{k,n}r_n
\end{equation}
Each factor return $b_k$ could be regarded as a portfolio weighted by $c_{k,n}$.\\
? The factor portfolio holdings are known a priori. The factor portfolio holdings ensure that the portfolio has unit exposure to the particular factor, zero exposure to every other factor, and minimum risk given those constraint. (not quiet understand yet)
\subsection{Factor Covariance Matrix}
\begin{itemize}
\item
weights over the past history
\item
bayesian priors on covariance
\item
forecasting variance conditional on recent events
\end{itemize}
\subsection{Specific Risk}
? By definition, the model cannot explain a stock's specific return $u_n$.\\
In general, specific risk is modeled by
\begin{equation}
u_n^2=S(t)[1+v_n(t)]
\end{equation}
where
\begin{equation}
\left(\frac{1}{N}\right)\sum_n u_n^2(t)=S(t)
\end{equation}
and
\begin{equation}
\left(\frac{1}{N}\right)\sum_n v_n(t)=0
\end{equation}
\section{Risk Analysis}
Then, results given by structural risk model could be easily extend to portfolio risk analysis. 
\begin{equation}
\v x_p=\v X^T\cdot \v h_p
\end{equation}
\begin{equation}
\sigma^2_p=\v x_p^T\cdot \v F \cdot \v x_p +\v h_p^T\cdot \gv \Delta \cdot \v h_p=\v h_p^T\cdot \v V \cdot \v h_p
\end{equation}
\subsection{Separation between common factors and specific factors}
\subsection{Separation between market risks and residual risk}
\subsection{Marginal Contribution}
\subsection{Factor Marginal Contribution}
\subsection{Sector Marginal Contribution(Rather ambiguous)}
\subsection{Attribution of Risk}
%----------------------------------------------------------------------------------------

%----------------------------------------------------------------------------------------
%	BIBLIOGRAPHY
%----------------------------------------------------------------------------------------

\phantomsection
\bibliographystyle{unsrt}
\bibliography{QuantRef}

%----------------------------------------------------------------------------------------
\end{addmargin}
\end{document}